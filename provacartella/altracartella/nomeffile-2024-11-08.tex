\documentclass[10pt]{article}

\usepackage[utf8]{inputenc}
\usepackage{geometry}
\usepackage{tabularx}
\usepackage{graphicx}
\usepackage[table]{xcolor}



%cambio misure della pagina
\geometry{a4paper,left=20mm,right=20mm,top=20mm}

\title{Preventivo Costi}
\date{A.A 2024/2025}

\renewcommand*\contentsname{Indice}
\begin{document}
%contenuti principali
\maketitle
\begin{center}
sevenbits.swe.unipd@gmail.com\\
\vspace{2mm}

\textbf{Registro modifiche}\\
\vspace{2mm}
\begin{tabular}{|l|l|l|l|l|}
\hline
\textbf{Versione} & \textbf{Data} & \textbf{Descrizione} & \textbf{Ruolo} & \textbf{Componente} \\
\hline
1.0 & 28/10/2024 & Stesura del preventivo & Scrittore & Gusella Manuel\\
& & & Scrittore & Trolese Leonardo\\
& & & Scrittore & Piva Riccardo\\
\hline
\end{tabular}
\end{center}
\newpage
\tableofcontents
\newpage
\section{Valutazione orari e costi}
\subsection{Costo orario per ruolo}
\begin{center}
\begin{tabular}{|c|c|c|}
\hline
\rowcolor{lightgray} \textbf{Ruolo} & \textbf{Costo Orario (\texteuro)} & \textbf{Ore per ruolo}\\
\hline
Responsabile & 30 & 65\\
Amministratore & 20 & 56\\
Analista & 25 & 68\\
Progettista & 25 & 120\\
Programmatore & 15 & 180\\
Verificatore & 15 & 160\\
\hline
\rowcolor{lightgray} & \textbf{Totale Costo (\texteuro)} & \textbf{Totale Ore}\\
\hline
& 12870 & 649\\
\hline
\end{tabular}
\end{center}

\subsection{Suddivisione delle ore per ruolo in base al membro}
\begin{center}
\rowcolors{2}{lightgray}{lightgray!50!white!50}
\begin{tabular}{|c|c|c|c|c|c|c|c|}
\hline
\textbf{Membro} & \textbf{Re} & \textbf{Am} & \textbf{An} & \textbf{Prj} & \textbf{Prg} & \textbf{Ver} & \textbf{Totale}\\
\hline
Cristellon & 9 & 8 & 10 & 18 & 27 & 23 & 95\\
Gusella    & 10 & 8 & 10 & 18 & 26 & 23 & 95\\
Peruzzi    & 9 & 8 & 9 & 17 & 26 & 23 & 92\\
Piva       & 10 & 8 & 10 & 17 & 27 & 23 & 95\\
Pivetta    & 10 & 9 & 10 & 17 & 26 & 23 & 95\\
Rubino     & 8 & 7 & 9 & 16 & 23 & 22 & 85\\
Trolese    & 9 & 8 & 10 & 17 & 25 & 23 & 92\\
\hline
\end{tabular}
\end{center}
Glossario dei termini in tabella:
\begin{itemize}
\item Re: Responsabile;
\item Am: Amministratore;
\item An: Analista;
\item Prj: Progettista;
\item Prg: Programmatore;
\item Ve: Verificatore.
\end{itemize}

\section{Suddivisione ruoli}
La suddivisione delle ore è stata fatta tenendo in considerazione del differente apporto che ciascun ruolo dà al progetto. Le riflessioni del gruppo sono state le seguenti:
\begin{itemize}
    \item Responsabile, essendo il nucleo del progetto, è necessario che abbia almeno un paio di ore a settimana (da moderare a causa dell'elevato costo), per garantirsi che il gruppo stia rispettando le scadenze;
    \item L'amministratore, come il responsabile, è un ruolo che è sempre presente, anche se questo ruolo non è troppo impegnativo a livello di tempo dato che il suo compito è quello di accertarsi che il team stia lavorando seguendo le regole prefissate e alcuni picchi quando ci sono dei deploy del prodotto;
    \item L'Analista, presente principalmente nella prima parte dello sviluppo del progetto, richiederà meno ore rispetto ad altri ruoli;
    \item Il Progettista è presente durante tutta la fase di sviluppo del progetto, e per questa ragione abbiamo scelto di attribuirgli molte ore, prevedendo che la fase di design sarà di importanza centrale per la realizzazione del capitolato;
    \item Il Programmatore, anche se presente solo nell'ultima parte di progetto, secondo noi coprirà molte ore, non eccessive, nella durata complessiva del progetto.
    \item Il Verificatore, poichè che è presente per tutta la durata del progetto, avrà un ruolo fondamentale e coprirà una buona parte del totale delle ore del progetto;
\end{itemize}
Le ore individuali sono state poi ripartite equamente e proporzionalmente fra ogni ruolo tenendo in considerazione la differenza di ore produttive dichiarate dai membri, e attribuendole a seconda di preferenze personali laddove una suddivisione completamente proporzionale non è stata possibile.

\section{Preventivo dei costi}
In base alla pianificazione effettuata per la suddivisione del monte orario nei vari ruoli presenti durante lo svolgimento del progetto, si prevede che il costo finale del progetto ammonti a 12870\texteuro . Costo verificabile nella prima tabella presente nel documento.
\section{Scadenza prevista}
Il gruppo prevede di poter consegnare il prodotto finale relativo al quarto capitolato "NearYou - Smart custom advertising platform" proposto dall'Azienda SyncLab entro il 21/03/2025.
\end{document}
